\documentclass{beamer}

\usepackage{hyperref}

\usepackage{array}
\usepackage{setspace} 
\usepackage{graphicx}
\usepackage[utf8]{inputenc}
\usepackage[croatian]{babel}
\usetheme{Singapore} 
\title[Problem upotpunjenja grafa do Hamiltonovog]{Problem upotpunjenja grafa do Hamiltonovog}
\author{Ante Grbi\'{c}, Hermina Petric Mareti\'{c}}
\institute{Prirodoslovno-matemati\v{c}ki fakultet - Matemati\v{c}ki odjel \\ Seminar za teorijsko ra\v{c}unarstvo}
\date{18. ožujka, 2013.}

\setstretch{1.25}

\begin{document}

\begin{frame}
\titlepage
\end{frame}

\begin{frame}{Sadr\v{z}aj}
\begin{enumerate}
\item Uvod
\item Promatrane heuristike
\item Implementacija
\item Usporedba rezultata
\item Zaklju\v{c}ak
\item Literatura
\end{enumerate}
\end{frame}

\section{Uvod}

\begin{frame}{Opis problema}

Neka je zadan neusmjeren, beste\v{z}inski graf $G$ = ($V$,$E$). Potrebno je dodati minimalan broj bridova $d \in \mathbb N$ grafu $G$ tako da graf postane Hamiltonov, odnosno da sadrži Hamiltonov ciklus.
Jasno je da je ovaj problem NP-te\v{z}ak jer rješava problem traženja Hamiltonovog ciklusa.\\
NP-potpunost problema traženja Hamiltonovog ciklusa dokazana je 1972. u $Reducibility\:Among\:Combinatorial\:Problems$, Karp. 

\end{frame}


\begin{frame}{Povijest problema i dosada\v{s}nji rezultati}

\begin{itemize}
\item Problem su neovisno predstavili Boesch, Chen i McCugh te Goodman i Hedetniemi 1970. godine. 
\item Objavili su algoritme koji u polinomijalnom vremenu konstruiraju optimalni pokriva\v{c} jednostavnim putevima za stablo.
\item Kundu je 1976. njihove rezultate pobolj\v{s}ao pokazav\v{s}i da se isti algoritam mo\v{z}e izvesti u linearnom vremenu.
\end{itemize}

\end{frame}

\begin{frame}{Povijest problema i dosada\v{s}nji rezultati}

2012. godine David Gamarnik i Maxim Sviridenko ra\v{c}unaju asimptotsko rje\v{s}enje problema za rijetke grafove $G$ = ($V$,$\frac{c} {n}$),  gdje je $c < 1$, a $\frac{c} {n}$ ozna\v{c}ava vjerojatnost s kojom se svaki brid mo\v{z}e pojaviti u grafu G. 
\end{frame}

\begin{frame}{Primjena problema}

\begin{itemize}
\item Dodijeljivanje frekvencija 
\item Optimizacija programskog koda
\item Distribuirani procesi (dodijeljivanje procesa distribuiranim procesorima)
\end{itemize}

\end{frame}

\section{Promatrane heuristike}
\begin{frame}{Algoritmi}

\begin{itemize}
\item Genetski algoritam - krizanje poretkom
\item Genetski algoritam - pohlepno kri\v{z}anje
\item Imunolo\v{s}ki algoritam
\item Mravlji algoritam 
\item Mravlji algoritam "Blizanac"

\end{itemize}

\end{frame}

\subsection{Genetski algoritmi}
\begin{frame}{Genetski algoritam}

Reprezentacija rje\v{s}enja: permutacije \v{c}vorova bez \v{c}vora broj 1.
Algoritam (po\v{c}inje se sa slu\v{c}ajno generiranom populacijom veli\v{c}ine $brJed$): \\
\begin{enumerate}
\item Ocijeni jedinke.
\item Odaberi dvije najbolje i sa\v{c}uvaj ih u $novaPop$. Roulette wheel selekcijom odaberi $brJed - 2$ jedinki i spremi ih u $novaPop$.
\item Metodom kri\v{z}anja poretka kri\v{z}aj odabrane jedinke.
\item Mutacijom koja je implementirana kao implementacija inverzije mutiraj jedinke u $novaPop$.
\item Vjerojatnost mutacije se s vremenom smanjuje.
\item Po\v{c}etnu populaciju zamijeni s $novaPop$. Nastavi algoritam dok nije zadovoljen uvjet prekida.
\end{enumerate}

\end{frame}

\begin{frame}{Kri\v{z}anje poretkom}

\includegraphics[width = 12cm, height = 7cm]{krizanje_poretkom.jpg} \\
\begin{center} \tiny (preuzeto iz (5)) \end{center}


\end{frame}


\begin{frame}{Mogu\'{c}e pobolj\v{s}anje genetskog algoritma}
Mogu\'{c}e pobolj\v{s}anje pohlepnim operatorom kri\v{z}anja, koji se kao ideja pojavljuje u problemu trgova\v{c}kog putnika. \\
Princip je taj da svi bridovi koji su zajedni\v{c}ki roditeljima (ciklusima)  postanu osnova njihove djece. Ostatak bridova se generira slu\v{c}ajno.
\end{frame}

\subsection{Imunološki algoritam}
\begin{frame}{Imunološki algoritam}

\begin{enumerate}
\item Stvorimo inicijalnu populaciju antitijela veličine d.
\item Svako antitijelo kloniramo k puta (sada imamo (k+1)*d antitijela).
\item Svaki klon prolazi proces hipermutacije.
\item Računamo afinitete starih antitijela i hipermutiranih klonova te gradimo novu populaciju od d najboljih.

\end{enumerate}

\end{frame}


\begin{frame}{Hipermutacija}
\begin{center}
\includegraphics[scale = 0.6]{mutacija.png}
\end{center}
\begin{center} \tiny (preuzeto iz (5)) \end{center}
\end{frame}

\subsection{Mravlji algoritmi}
\begin{frame}{Mravlji algoritam}

\begin{enumerate}
\item Napravimo kopiju matrice prijelaza koju \'{c}e mravi mijenjati.
\item Svi mravi kre\'{c}u iz istog vrha i odabiru bridove s vjerojatno\v{s}\'{c}u $\frac{c_n}{\sum{c_n}}  $ ako postoje slobodni bridovi, odnosno nasumi\v{c}no dodaju novi brid ako ne postoji nastavak puta.
\item Mravi evaluiraju svoja rje\v{s}enja i uspore\dj{}uju ih s dosad najboljim putevima.
\item Najbolji mrav ostavlja feromone na svim ve\'{c} postoje\'{c}im bridovima (i samo njima).
\item Trag na cijelom grafu slabi. 

\end{enumerate}

\end{frame}

\begin{frame}{Mravlji algoritam - "blizanac"}

\begin{itemize}
\item Pohlepna modifikacija mravljeg algoritma
\item Mravi kre\'{c}u iz vrha s najmanje susjeda
\item Kad do\dj{}u do kraja puta, javlja se blizanac koji se nalazi na po\v{c}etku puta i nastavlja put iz po\v{c}etne to\v{c}ke
\item Kad blizanac do\dj{}e do kraja puta, dodaje brid prema onom vrhu koji ima najmanje susjeda i postupak se nastavlja.
 
\end{itemize}

Blizanac opravdava pohlepnost algoritma!

\end{frame}


\begin{frame}{Dokaz (1)}

Neka je $x$ = $x_{1}, x_{2},..., x_\text{n}$ jedan optimalan ciklus i $ x_\text{k} = a_{1}$ element s minimalnim brojem susjeda.
Mrav može kopirati put $x_\text{k},...,x_\text{k+l}$ sve dok su ti bridovi postojeći u grafu. Neka je $a_{1}...a_\text{l+1}$ = $x_{k}...x_{k+l}$. Ako $a_\text{l+1}$ više nema susjeda, dolazi blizanac, a ako ima, mrav bira susjeda i opet kopira sve postojeće bridove iz $x$ u jednom smjeru. Dalje induktivno. \\
Primijetimo da ovim postupkom nismo poremetili optimalnost ciklusa (nismo dodali nijedan brid, a razdvojenih dijelova ima manje ili jednako kao i prije).

\end{frame}

\begin{frame}{Dokaz (2)}

Sada dolazi blizanac. On može kopirati sve postojeće i neposjećene bridove prije $x_\text{k}$, neka su to $x_\text{k-1}, ..., x_\text{k-s}$. Nakon toga ponavljamo već opisani postupak dok blizanac ne dođe do vrha bez susjeda. \\
Mrav skače na neiskorišteni vrh s najmanjim brojem susjeda i ponavljamo gornje argumente. \\
Poanta je u tome da ako mrav krene od "krivog" vrha ili skoči na njega, blizanac lako ispravi problem.

\end{frame}

\begin{frame}{Moguće poboljšanje genetskog i imunološkog algoritma}

\begin{itemize}
\item Početnu populaciju ne generiramo nasumično, već na neki educirani način
\item Koristimo mrave kako bismo generirali početne jedinke
\item Drastično poboljšanje, pogotovo za genetski algoritam
 
\end{itemize}
\end{frame}

\section{Implementacija}

\begin{frame}{Napomena}

Promatramo samo grafove oblika $G$ = ($V$,$\frac{c} {n}$), gdje $\frac{c} {n}$ ozna\v{c}ava vjerojatnost s kojom se svaki brid mo\v{z}e pojaviti u grafu G po uzoru na ranije spomenuti članak autora Gamarnik, Sviridenko: $Hamiltonian\:Completions\:of\:Sparse\:Random\:Graphs$.

\end{frame}
\begin{frame}{Implementacija}
\begin{itemize}
\item Svi algoritmi su implementirani u MATLAB-u\\
\item Grafovi su spremljeni pomoću matrica incidencije\\
\item Matrice su čuvane u kompresiranom sparse formatu\\
\item Funkcije za I/O operacije sa sparse matricama su preuzete sa \url{http://math.nist.gov/MatrixMarket/mmio/matlab/mmiomatlab.html}\\ 
\end{itemize}
\end{frame}

\begin{frame}{Korisni\v{c}ko su\v{c}elje}
\begin{center}
\includegraphics[scale = 0.28]{gui.png}
\end{center}
\end{frame}

\section{Usporedba rezultata}
\begin{frame}{Opis testiranja}
\begin{itemize}
\item Testiranja svih algoritama su provedena na 30 nasumično generiranih grafova od 200 vrhova i 2 nasumično generirana grafa od 1000 vrhova.
\item Sva testiranja su provedena tri puta nad istim grafovima s jednakim parametrima te rezultati prikazuju usrednjenje.
\item Bridovi na svim grafovima su generirani s vjerojatnošću 1/n.
\item Posebno su provedena testiranja za Mravlji algoritam "Blizanac" na nasumično generiranim grafovima od 2000, 4000, 8000  i 10000 vrhova.
\end{itemize}

Sva testiranja obavljena su na računalu s procesorom Intel Core i5, 2,3GHz i 6GB RAM memorije.
\end{frame}


\begin{frame}{200 vrhova}
\begin{center}
\includegraphics[scale = 0.55]{par2.png}\\
\tiny
Elitizam: 2\\
Trajanje izvođenja: 52.63 sekunde\\
Najbolji rezultat: 127.667\\
\end{center}
\end{frame}

\begin{frame}{200 vrhova}
\begin{center}
\includegraphics[scale = 0.55]{par4.png}\\
\tiny
Elitizam: 4\\
Trajanje izvođenja: 47.5709 sekunde\\
Najbolji rezultat: 124\\
\end{center}
\end{frame}


\begin{frame}{200 vrhova}
\begin{center}
\includegraphics[scale = 0.55]{par6.png}\\
\tiny
Elitizam: 6\\
Trajanje izvođenja: 42.68 sekunde\\
Najbolji rezultat: 122\\
\end{center}
\end{frame}


\begin{frame}{200 vrhova}
\begin{center}
\includegraphics[scale = 0.55]{par8.png}\\
\tiny
Elitizam: 8\\
Trajanje izvođenja: 40.01 sekunde\\
Najbolji rezultat: 122.333\\
\end{center}
\end{frame}

%\begin{frame}{200 vrhova}
%\begin{center}
%\includegraphics[scale = 0.4]{Gen200.png}\\
%Trajanje izvođenja: 52.63 sekunde\\
%Najbolji rezultat: 125.66\\
%\end{center}
%\end{frame}

\begin{frame}{200 vrhova}
\begin{center}
\includegraphics[scale = 0.4]{Im200.png}\\
Trajanje izvođenja: 100.11 sekundi\\
Najbolji rezultat: 118.33\\
\end{center}
\end{frame}

\begin{frame}{200 vrhova}
\begin{center}
\includegraphics[scale = 0.4]{Mr200.png}\\
Trajanje izvođenja: 24.41 sekunde\\
Najbolji rezultat: 118\\
\end{center}
\end{frame}

\begin{frame}{200 vrhova}
\begin{center}
\includegraphics[scale = 0.4]{MrBl200.png}\\
Trajanje izvođenja: 1.17 sekundi\\
Najbolji rezultat: 116\\
\end{center}
\end{frame}

\begin{frame}{200 vrhova - usporedba rezultata}
\begin{center}
\includegraphics[scale = 0.45]{prvih10.png}\\
\end{center}
\end{frame}

\begin{frame}{200 vrhova - usporedba rezultata}
\begin{center}
\includegraphics[scale = 0.45]{drugih10.png}\\
\end{center}
\end{frame}

\begin{frame}{200 vrhova - usporedba rezultata}
\begin{center}
\includegraphics[scale = 0.45]{zadnjih10.png}\\
\end{center}
\end{frame}

\begin{frame}{1000 vrhova}
\begin{center}
\includegraphics[scale = 0.4]{Gen1000.png}\\
Trajanje izvođenja: 5136.6 sekundi\\
Najbolji rezultat: 644\\
\end{center}
\end{frame}

\begin{frame}{1000 vrhova}
\begin{center}
\includegraphics[scale = 0.4]{Im1000.png}\\
Trajanje izvođenja: 3045.6 sekundi\\
Najbolji rezultat: 590\\
\end{center}
\end{frame}

\begin{frame}{1000 vrhova}
\begin{center}
\includegraphics[scale = 0.4]{Mr1000.png}\\
Trajanje izvođenja: 2399.5 sekunde\\
Najbolji rezultat: 611\\
\end{center}
\end{frame}

\begin{frame}{1000 vrhova}
\begin{center}
\includegraphics[scale = 0.4]{MrBl1000.png}\\
Trajanje izvođenja: 24.51 sekundi\\
Najbolji rezultat: 582\\
\end{center}
\end{frame}


\begin{frame}{1000 vrhova - usporedba rezultata}
\begin{center}
\includegraphics[scale = 0.45]{Zajedno1000.png}\\
\end{center}
\end{frame}

\begin{frame}{Rezultati mravljeg algoritma "Blizanac"}
\begin{center}
\includegraphics[scale = 0.45]{boxplot.png}\\
\end{center}
\end{frame}


\begin{frame}{2000 vrhova}
\begin{center}
\includegraphics[scale = 0.45]{1.png}\\
Trajanje izvođenja: 99.7 sekundi\\
\end{center}
\end{frame}

\begin{frame}{4000 vrhova}
\begin{center}
\includegraphics[scale = 0.45]{2.png}\\
Trajanje izvođenja: 567.4 sekundi\\
\end{center}
\end{frame}

\begin{frame}{6000 vrhova}
\begin{center}
\includegraphics[scale = 0.45]{3.png}\\
Trajanje izvođenja: 1786.2 sekunde\\
\end{center}
\end{frame}

\begin{frame}{8000 vrhova}
\begin{center}
\includegraphics[scale = 0.45]{4.png}\\
Trajanje izvođenja: 4147.5 sekunde\\
\end{center}
\end{frame}

\begin{frame}{10000 vrhova}
\begin{center}
\includegraphics[scale = 0.45]{5.png}\\
Trajanje izvođenja: 8796.2 sekunde\\
\end{center}
\end{frame}

\section{Zaključak}
\begin{frame}{Zaključak}
\begin{itemize}
\item Genetski algoritam se pokazuje presporim i nedovoljno točnim za pristup ovom problemu.
\item Imunološki algoritam nudi blago poboljšanje u odnosu na genetski u aspektu brzine i osjetno poboljšanje u aspektu točnosti.
\item Mravlji algoritam vrlo brzo pronađe solidno rješenje, ali vrlo teško konvergira u točnije.
\item Daleko najbolje rezultate daje modifikacija mravljeg algoritma koja vrlo brzo konvergira u najbolja pronađena rješenja.
\end{itemize}
\end{frame}

\begin{frame}{Zaključak}
Daljnja poboljšanja:
\begin{itemize}
\item Implementacija vlastite sparse strukture prilagođene potrebama naših algoritama.
\item Paralelizacija algoritama.
\item Eventualno bolje prilagođeni operatori križanja za genetski algoritam.
\end{itemize}
\end{frame}

\section{Literatura}
\begin{frame}{Literatura}
\tiny
\begin{enumerate}
\item Darrell Whitley, Nam-Wook Yoo: Modeling Simple Genetic Algorithms for Permutation Problems, C.Sc. Department, Colorado State University
\item David Gamarnik, Maxim Sviridenko: Hamiltonian Completions of Sparse Random Graphs, MIT, 2012.
\item D. S. Franzblau, A.Raychaudhuri: Optimal Hamiltonian completions and path covers for trees, and a reduction to maximum flow; Department of Mathematics, CUNY, College of Staten Island, USA, 1999.
\item \url{http://en.wikipedia.org/wiki/Hamiltonian_completion}
\item Marko \v{C}upi\'{c}: Prirodom inspirirani optimizacijski algoritmi. Metaheuristike; FER, Sveu\v{c}ili\v{s}te u Zagrebu
\item \url{http://math.nist.gov/MatrixMarket/mmio/matlab/mmiomatlab.html}

\end{enumerate}

\end{frame}


\end{document}
